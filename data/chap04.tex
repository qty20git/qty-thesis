% % !TeX root = ../thuthesis-example.tex

\chapter{结论与展望}

本文建立了包含507中可能工艺路径的卢非酰胺连续合成过程超结构优化模型,并在不确定条件存在下通过两阶段可调鲁棒优化方法找到了最优路径选择,并设计了预期年成本投入1.2千万美元的最优合成过程。

本文针对卢非酰胺药物API分子的上游合成设计缺少概念设计、综合与不确定性优化结合的研究这一问题,利用已有的对卢非酰胺连续合成模式生产有应用潜力的工艺路径建立了基于超结构优化的过程综合与设计模型,在应对三种生产不确定性并存的情景下,采用了两阶段可调鲁棒优化方法建立了同时对工艺路径选择进行筛选和进行过程设计优化的两阶段模型。该模型的决策对生产不确定性有充分的应对能力,且可以识别出在不确定条件下更优的超结构决策,提供更大的生产设计空间。

为解决两阶段超结构优化模型中存在的混合整数追索问题,本文提出了一种改进的基于列与约束生成算法的求解流程,可以在不引入决策规则前提下精确求解含有混合整数追索的两阶段可调鲁棒优化问题。该计算流程的提出为两阶段模型的建立拓展了应用与研究的场景,使基于超结构优化的化工过程综合与设计优化模型拥有了更加广阔丰富的应用前景,

然而,与本文研究目标相关的课题仍然还有很多值得进一步探讨的内容。首先,从求解方法上来看,当前算法在规模不算太大的优化模型中尚有可接受的计算效率表现,然而对于更一般的化工过程设计问题,其求解效率仍然太低,而且灵活性不足,三层问题结构下算法流程的应用便利性和可扩展性也受到较大限制。未来更有效率的分解与剪枝策略,或基于学习策略的定制切平面方案与单纯应用列与约束生成算法的框架结合可能会提供更加高效与可扩展的含有混合整数追索的ARO求解算法。

其次,在模型方法上,本文对于卢非酰胺这一特定化学品生产目标的工艺综合与设计,问题本身存在一定场景局限性,在实际高端化学品生产工业中,往往面临着多条产线的并行、调度、联合生产,还有许多工业化生产场景的细节未能充分考虑,因此本文的模型相比实际落地还有较远的距离,简化程度仍然较高。为了更好地连接概念设计阶段与项目方案设计阶段,更详细的工艺细节开发与更全面的经济评估是下一步亟待开展的工作。






% 模板支持 BibTeX 和 BibLaTeX 两种方式处理参考文献。
% 下文主要介绍 BibTeX 配合 \pkg{natbib} 宏包的主要使用方法。


% \section{顺序编码制}

% 在顺序编码制下,默认的 \cs{cite} 命令同 \cs{citep} 一样,序号置于方括号中,
% 引文页码会放在括号外。
% 统一处引用的连续序号会自动用短横线连接。

% \thusetup{
%   cite-style = super,
% }
% \noindent
% % \begin{tabular}{l@{\quad$\Rightarrow$\quad}l}
%   \verb|\cite{zhangkun1994}|               & \cite{zhangkun1994}               \\
%   \verb|\citet{zhangkun1994}|              & \citet{zhangkun1994}              \\
%   \verb|\citep{zhangkun1994}|              & \citep{zhangkun1994}              \\
%   \verb|\cite[42]{zhangkun1994}|           & \cite[42]{zhangkun1994}           \\
%   \verb|\cite{zhangkun1994,zhukezhen1973}| & \cite{zhangkun1994,zhukezhen1973} \\
% \end{tabular}


% 也可以取消上标格式,将数字序号作为文字的一部分。
% 建议全文统一使用相同的格式。

% \thusetup{
%   cite-style = inline,
% }
% \noindent
% \begin{tabular}{l@{\quad$\Rightarrow$\quad}l}
%   \verb|\cite{zhangkun1994}|               & \cite{zhangkun1994}               \\
%   \verb|\citet{zhangkun1994}|              & \citet{zhangkun1994}              \\
%   \verb|\citep{zhangkun1994}|              & \citep{zhangkun1994}              \\
%   \verb|\cite[42]{zhangkun1994}|           & \cite[42]{zhangkun1994}           \\
%   \verb|\cite{zhangkun1994,zhukezhen1973}| & \cite{zhangkun1994,zhukezhen1973} \\
% \end{tabular}



% \section{著者-出版年制}

% 著者-出版年制下的 \cs{cite} 跟 \cs{citet} 一样。

% \thusetup{
%   cite-style = author-year,
% }
% \noindent
% \begin{tabular}{@{}l@{$\Rightarrow$}l@{}}
%   \verb|\cite{zhangkun1994}|                & \cite{zhangkun1994}                \\
%   \verb|\citet{zhangkun1994}|               & \citet{zhangkun1994}               \\
%   \verb|\citep{zhangkun1994}|               & \citep{zhangkun1994}               \\
%   \verb|\cite[42]{zhangkun1994}|            & \cite[42]{zhangkun1994}            \\
%   \verb|\citep{zhangkun1994,zhukezhen1973}| & \citep{zhangkun1994,zhukezhen1973} \\
% \end{tabular}

% \vskip 2ex
% \thusetup{
%   cite-style = super,
% }
% 注意,引文参考文献的每条都要在正文中标注
% \cite{zhangkun1994,zhukezhen1973,dupont1974bone,zhengkaiqing1987,%
%   jiangxizhou1980,jianduju1994,merkt1995rotational,mellinger1996laser,%
%   bixon1996dynamics,mahui1995,carlson1981two,taylor1983scanning,%
%   taylor1981study,shimizu1983laser,atkinson1982experimental,%
%   kusch1975perturbations,guangxi1993,huosini1989guwu,wangfuzhi1865songlun,%
%   zhaoyaodong1998xinshidai,biaozhunhua2002tushu,chubanzhuanye2004,%
%   who1970factors,peebles2001probability,baishunong1998zhiwu,%
%   weinstein1974pathogenic,hanjiren1985lun,dizhi1936dizhi,%
%   tushuguan1957tushuguanxue,aaas1883science,fugang2000fengsha,%
%   xiaoyu2001chubanye,oclc2000about,scitor2000project%
% }。
