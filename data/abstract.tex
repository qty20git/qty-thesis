% !TeX root = ../thuthesis-example.tex

% 中英文摘要和关键字

\begin{abstract}
  随着化学工业的高端化、绿色化发展,高附加值化学品的生产正在向连续化工艺集成转变。以新型抗癫痫三唑衍生物卢非酰胺为例,其传统的间歇生产方式存在产品质量不易控制、产能难以扩大、环境风险高等内在缺陷,因此其连续合成受到工艺开发者的广泛关注。随着流动微反应器工艺的研究成熟,卢非酰胺合成过程中的危险叠氮化物中间体累积所带来的安全风险也将被规避。因此,运用系统工程方法,针对卢非酰胺的连续合成过程进行优化设计具有大幅降低工艺开发的风险和成本,综合考虑工艺过程的可操作性,经济效益和环境效应的意义。

  基于从现有文献及专利库中整理的共507种采用流动微反应器的卢非酰胺连续合成路线,本论文在综合考虑多种生产过程面临的不确定性的情况下,建立了合成方案设计超结构优化模型,并采用可调鲁棒优化以同时考虑对不确定性应对能力最优的合成路径筛选与过程设计优化。在第二阶段的过程设计模型中,存在微反应器设备变量为整数,传统的两阶段可调鲁棒优化求解算法无法精确求解。本论文提出了一种改进的基于列与约束生成算法的求解流程,实现了对所建立的可调鲁棒优化模型的高效最优化求解,并得出了潜在的最优合成路线设计,为卢非酰胺的进一步工艺研发提供了指导。

  % 关键词用“英文逗号”分隔,输出时会自动处理为正确的分隔符
  \thusetup{
    keywords = {过程设计, 卢非酰胺, 超结构优化, 可调鲁棒优化, 列与约束生成算法},
  }
\end{abstract}

\begin{abstract*}
  With the advancement of the chemical industry towards high-end and green development, the production of high value-added chemicals is shifting towards the integration of continuous chemical processes. The continuous synthesis of rufinamide, a novel anti-epileptic triazole derivative, whose traditional batch production method suffers from inherent flaws such as difficulty in controlling product quality, limited production capacity, and high environmental risks, has gained widespread attention by chemical process developers. As the research on flow microreactor technology matures, the safety risks associated with the accumulation of hazardous diazonium intermediates during the synthesis of rufinamide could be mitigated. Therefore, through the application of systems engineering methods, optimizing the continuous synthesis process of rufinamide aims to significantly reduce the risks and costs of process development, while considering the operational feasibility, economic benefits, and environmental impacts.
  
  Based on the 507 continuous synthesis routes of rufinamide using flow microreactor technology collected from existing literature and patent databases, this study has developed a superstructure optimization model for the designing of synthesis process, and has applied adjustable robust optimization under considerations of multiple uncertainties faced in various production scenarios, to simultaneously optimize the selection of synthetic routes and process design that best addresses the uncertainties. In the second stage of the process design model, where the number of microreactor equipment is set as integer variable, traditional two-stage adaptable robust optimization algorithms fail to provide exact solutions. To address this issue, an improved solution strategy based on the column-and-constraint generation algorithm has been proposed, enabling efficient and optimal solutions to the adjustable robust optimization model established. The potential design of the optimal synthesis route has been identified, guiding further process development of rufinamide.

  % Use comma as separator when inputting
  \thusetup{
    keywords* = {process design, rufinamide, superstructure optimization, adjustable robust optimization, column-and-constraint generation algorithm},
  }
\end{abstract*}
