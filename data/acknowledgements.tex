% !TeX root = ../thuthesis-example.tex

\begin{acknowledgements}
  衷心感谢导师袁志宏老师的指导、帮助与关心。袁老师在我对PSE学科的入门与科研探索方面给予了非常大的支持与鼓舞,并且资助我参加重要的国际学术会议参与交流。袁老师在PSE专业方面深厚的知识积累与精益求精、学习进取的精神是我永远前进的动力和学习的榜样。

  感谢同课题组的2018级博士毕业生章立峰师兄,2019级博士生杨雯惠师姐、胡昊阳师兄、梁润喆师兄,2020级博士生葛丛钦师兄,2022级硕士生温唯谷师兄、韩月衡师兄等前辈们为我科研、课题和生活等方面提供的建议与帮助,他们不仅是我科研道路上学习的对象,也是生活中的挚友,课题组团结向上、严肃活泼、互帮互助的氛围让我对科研有了更强的动力和信心。

  感谢清华大学化学工程系过程系统工程研究所的赵劲松老师、邱彤老师等为本课题提出的打磨建议。感谢研究所其他课题组的所有硕博研究生朋友们的陪伴与鼓舞,让我感到身处一个温暖的大家庭。

  感谢清华大学探微书院为我本科阶段提供的良好的学习与科研探索的平台,这是一个富有理想主义色彩和集体凝聚力、热情积极、活泼向上的集体。我要感谢书院的刘铮老师、卢滇楠老师、邢新会老师、张紫千老师和罗嵘老师对我本科期间的关心和帮助,感谢可爱的辅导员们,感谢探微书院学生会(筹)及第一届团学组织大家庭给我留下的美好回忆。

  感谢探微-化01班的兄弟姐妹们、书院党委探微0党支部的全体党员同志们、探微书院第一届学生会、“思源骨干计划”第十七期以及我每一个微信群聊中的朋友们给我带来的陪伴与快乐,感谢室友在生活方面的包容与帮助,以及所营造的充满欢笑的寝室环境。

  感谢父母、亲友,以及所有以上未涵盖但对我支持、关心、默默帮助的人们。

  感谢清华大学TUNA协会提供的论文模板。

\end{acknowledgements}
