% !TeX root = ../thuthesis-example.tex

\chapter{引言}

\section{研究背景及意义}

以药物活性分子(Active Pharmaceutical Ingredient, API)为代表的高附加值化学品的生产模式可以被宏观地分为间歇合成(Batch Synthesis)模式与连续合成(Continuous Synthesis)模式。前者通常代表生产者需要一套满足各类工序及单元操作要求的间歇式设备,根据特定的产品目标按照一定工序分批次地将生产所需物料投放到设备中,获得每步工序的产出之后再人为转移到下步工序的设备中。间歇合成模式的发展与成熟较早,生产过程与设备设计相对简单,在人力充足的条件下对自动控制要求很低,因此在过去被广泛应用。然而,随着人们对环境保护、可持续发展与高质量发展的日渐关注,间歇合成暴露出一系列缺陷:物料输运、能量利用和工况转换的效率低,难以规模化生产;中间产物的运输、存储带来额外的成本与安全风险;各批次的产品质量难以保持稳定;污染排放严重、环境影响较大等等\cite{plumb2005,singh2012}。

相反,连续合成模式则需要生产工艺开发者根据预期生产目标和规划定制相应的连续生产设备与系统,以达到各步工序和反应不间断地进行,物料在各设备单元间直接流动,尽可能减少人力与外界干扰的介入\cite{cole2018}。相比间歇合成,连续合成过程由于操作状态稳定,干扰较少,通常具有更高的物质与能量输送效率而拥有更稳定的产品质量、更高的生产效率和更大的生产规模,同时由于物流的隔离、循环利用和统一处理,连续合成模式下生产过程的环境影响也能够有效地控制。目前,在药物合成领域,一些西方发达国家已经向连续化合成的生产模式转变,并取得了较好的发展前景与较强的市场优势\cite{markarian2021},然而在发展中国家,连续合成模式的推广应用面临前期开发投入较高、工艺设计复杂、缺少工业经验积累等瓶颈,尚未成为主流的生产模式,也由此极大地限制了行业的发展。因此,针对高附加值化学品的连续合成过程在设计初期阶段,采用过程综合与优化等过程设计(Process Design)方法开展对可行工艺路线、生产流程与生产设备方案的概念设计(Conceptual Design)具有重要的价值,可以允许生产工艺开发者从较多尚停留在实验室或小试开发阶段的合成工艺、技术路线中,根据生产目标、成本控制、应对不确定性的风险能力等综合因素考量,识别出最优的合成工艺组合流程,并给出相应流程所需要的生产条件、设备与设计规模,由此可以极大地减轻生产开发的筛选难度和成本投入,缩短预期工艺开发周期\cite{smith1995}。随着计算机优化理论的成熟,计算机辅助工艺工程(Computer-Aided Process Engineering, CAPE)在化工过程设计中地位不断升高,研究者通过建立化工工艺流程的数学优化模型可以提供更加理想的决策方法,为概念设计提供科学便捷的途径。

随着对过程设计方法的深入研究,不确定性分析(Uncertainty Analysis)成为工艺设计与优化研究者日益关切的内容。对于连续合成过程的设计而言,广泛存在的不确定性因素将对过程设计中的决策问题产生关键影响,这些不确定性来自各个方面,有对某一过程的机理掌握不明确导致确定的数学形式无法准确刻画系统某一部分的输入与输出特性,即模型不确定性(Model Uncertainty),比如反应热力学、动力学模型等\cite{ulas2004};也有因为控制因素或者外部干扰使得生产设计问题的参数发生波动,即过程不确定性(Process Uncertainty),如组成、流量、温度或供应量、需求量、价格等\cite{pisti1995}。连续合成模式往往具有开停车时间较长、高度依赖稳态操作条件、内部流动因素复杂的特征,因此一些设计决策面临应对不确定性的灵活程度较低的问题,这将威胁到概念设计决策与方案的优越性甚至可行性,因此需要针对性地研究。然而,针对高附加值化学品的连续合成过程的概念设计与不确定性分析方法的结合尚未有充分的研究,因此,本文开展的方法研究与案例应用具有一定的启发价值。

本文开展案例研究所针对的合成目标是卢非酰胺(Rufinamide),一种具有抗癫痫作用的三唑类酰胺衍生化合物,已经成为在美国药监局(U.S. Food \& Drug Administration, FDA)监管下批准上市的API种类。以Banzel成品药为例,其对2023年国家卫健委颁布的第二批罕见病中的Lennox-Gastaut综合征(LGS)有针对性治疗作用,混悬液售价高达3.90美元每毫升,具有非常重要的药品战略地位,同时具有较高的附加价值。由常见的平台化学品 {2,6-}二氟苯甲醇作为前驱体出发,合成卢非酰胺成品需经历三个主要的反应步骤,分别是卤化、重氮化与重氮-炔烃环加成(Azide-alkyne Cycloaddition, AAC)反应。三个步骤均为传统的有机合成反应,溶剂、辅助试剂的需求量与消耗量较大,间歇设备相对简单易得,因此传统工艺中均采用间歇合成模式制备。然而,该工艺的第二步反应所产生的易爆有毒有机中间体2,6-二氟苯基重氮甲烷具有较高的安全风险,还面临着其他传统间歇方式的弊端。目前,采用流动微反应器(Flow Micro-reactor)工艺的卢非酰胺连续化合成在实验室层面的工艺研发已经有所报道\cite{borukhova2016},且有机工艺工作者对各个步骤的反应条件进行了丰富的研发与改进,可供选择的工艺路线非常多样。然而,不同路径的选择在卢非酰胺连续合成的过程设计中有不同的优势与代价,目前只有针对其中特定单元的工艺决策开展过程模拟与优化的工作\cite{diab2018},尚未存在工作将卢非酰胺三步合成步骤的各种可能路线进行整合并进行综合分析,选取在经济效益和应对不确定性能力方面最有潜在优势的路径,本文尝试进行这方面的工作,并为卢非酰胺连续合成工艺开发者提供更具框架性的指导。


\section{基于超结构优化的过程设计}
\subsection{过程设计方法总结}
过程设计是一个综合了化工工艺开发、产品研发、生产建造与调度规划的各个方面的总体概念,拥有系统化的方法供研究者与生产者应用于不同的生产场景、工业过程和具体行业,其中针对开发早期或生产规划阶段的最重要的技术是过程综合(Process Synthesis),也被认为是化工过程设计的核心之一\cite{smith1995, barnicki2004}。
过程综合包含流程综合(Flowsheet Synthesis)、设备网络调优、过程模拟与优化、经济评估等多个分支,通常是在基于物料与能量衡算的基础上运用适当的模型方法,在根据特定化工生产的要求和限制对可能的方案进行定量的评价与对比\cite{osti_293030}。
在过程设计方法发展初期,为了应对多个不同子系统之间的关联结构的搜索空间(Search Space)复杂度急剧升高的问题,研究了启发式结构策略(Heuristic Structuring)\cite{masso1969},在上世纪,启发式方法广泛应用于流程综合与公用工程网络调优这类涉及到大量子系统关联的问题\cite{papoulias1983,westerberg2004},可以大幅度地降低搜索调优的难度,这是由于启发式方法通常可以更好地将经验知识与实际问题相结合,避免了大量盲目试验。

随着大规模混合整数规划的理论发展,过程设计中的主要问题逐渐由数学规划问题所描述并取得很好的求实际效果,物料守恒、能量守恒、操作逻辑和单元系统的响应特性常常作为约束,而各类过程的决策与设计常用0-1二元变量和连续变量所描述,并构造符合设计者预期的优化目标函数。研究者先后提出了层次分解模型(Hierarchical Decomposion)和超结构综合模型(Superstructure Synthesis),后者可以更高效地应用于有大量潜在结构可供选择的情景,而更适用于包括本文研究在内的大量复杂过程设计问题\cite{westerberg2004, mencarelli2020,ryu2020}。

基于超结构综合模型的过程设计方法最早由Umeda等人提出,所谓超结构即包含一个过程系统所有可能的结构组合的选择的集合,开展具体过程设计问题的超结构优化通常遵循三步“假设——转化——求解”,即假设一个已经包含了所有纳入考虑的选择的超结构,转化为合适的数学规划模型并求解该数学规划\cite{umeda1972},本文也将根据这样的步骤范式展开研究。


\subsection{超结构优化有关工作}

基于超结构的设计方法允许设计者根据需求同时考虑多个不同的决策目标,比如同时考虑成本和灵活性的工艺水网络的集成\cite{ahmetovic2011};或同时考虑多个不同决策场景,比如同时考虑反应器设计与公用工程网络的复杂设备网络优化\cite{madenoor2018},这也是超结构优化更受到决策者青睐的因素。针对以经济性评估为主要目标函数,综合考虑多种可供选择的转化途径、技术或设备在物料转化网络中的集成组合,已有研究者在废溶剂回收利用\cite{chea2020}、微藻生物质利用\cite{gong2017}、丙烯和液态燃料生产\cite{yuan2016}等化学工业场景开展了基于超结构优化的过程设计工作。在化学品合成与生产的方面,超结构优化更显示出其独特的优越性,例如Matsunami等在固体药品下游的生产过程进行了基于超结构优化的过程综合与经济评估,为决策者提供了满足生产需求的最优结构\cite{matsunami2020},然而相应地,超结构优化方法在上游药品合成阶段的应用尚未有研究应用的案例。

用于过程设计的超结构优化模型有多种类别,在化工过程设计中最常见的是状态-任务网络(State-Task Network,STN),广泛应用于分离序列、换热网络、间歇反应或连续反应过程的建模优化\cite{kondili1993,yeoman1999}。近年以来,超结构的内容随着应用到更广泛的工业决策场景中而种类更加丰富,尤其在合成过程相关的研究中,节点常常代表某一中间产物或某一物流状态,而连接关系代表某种生产路径或者技术选择,这种建模方式已经在能源化工领域得到了一定应用\cite{manuelrestrepo2021}。以STN为代表的超结构优化模型的节点与连线关系往往蕴含了丰富的逻辑关联,因此Raman与Grossmann发展了广义析取规划(Generalized Disjunctive Programming, GDP)\cite{raman1994},从而系统地将超结构模型转化为含二元变量的MILP/MINLP问题进行求解。因此,复杂过程系统的超结构优化设计的挑战转化为了求解大规模MILP/MINLP问题的挑战\cite{turkey1996}。
幸运的是,通过大量过程优化理论研究者的努力,一部分的MILP/MINLP问题已经有了较为高效的基于主问题-子问题分解的求解算法,如广义Benders分解算法\cite{geoffrion1972},外部近似算法(Outer Approximation Algorithm)与等式松弛(Equality Relaxation)策略\cite{duran1986,kocis1988}等。采用 {BARON} 等商用求解器\cite{kilinc2018}可以直接高效地通过模型直接自适应地运用上述分解算法求解。

\section{不确定性优化方法}

连续合成过程存在的大量不确定性有时对优化设计问题的性质造成严重的影响,因此针对过程设计问题中面临的不确定性,学界探索了各类不确定性优化方法,例如柔性分析(Flexibility Analysis, FA)、随机规划(Stochatic Programming, SP)、鲁棒优化(Robust Optimization, RO)等等。不确定性呈现在数学模型中通常表现为除决策变量的参数由定值转化为在某个集合上任意取值或按照某一概率分布取值,该集合被称为不确定集(Uncertainty Set),不确定集的刻画与处理是不确定性优化领域研究的热点之一。

\subsection{不确定性优化总结}
不确定性优化处理的问题通常可以总结为在一部分不确定参数$\symbfit u\in\mathbb{U}$在一定的波动状况或一定取值范围下,决策者仍然希望获取较优的决策目标值与具体的决策变量,因此,如何界定不确定参数的变化范围,以及在解的最优性(Optimality)和对不确定参数的保守性(Conservatism),即在不确定集内改变不确定参数的取值(称为实现值,Realization)当前决策仍然可行的性质,此两者之间如何达到平衡,是不同不确定性优化方法的关键区别所在。由于早期化工过程设计与综合问题中,研究者更多地关注特定的流程设计能否在多种情景或条件变化下保持良好运行,常常假设某一具体的工艺设备或者操作可以人为调整的情况,因此在模型建立时把决策变量分为结构变量$\symbfit y\in\{0,1\}^m$,以表示超结构的选取;状态变量$\symbfit x\in\mathbb{X}$,以表示物理约束中需要的性质或者状态量;设计变量$\symbfit d\in\mathbb{D}$,以表示在生产场景或工艺设计中确定之后不可调整的决策;控制变量$\symbfit z\in\mathbb{Z}$,以表示在生产过程中可以人为调整以应对不确定性的决策\cite{halemane1983,swaney1985,pisti1995}。基于此,Swaney和Grossmann提出了柔性(Flexibility)概念\cite{swaney1985},并衍生出一系列柔性分析模型与相应的求解问题。柔性是过程设计与优化领域最早能够全面描述设计决策应对不确定性能力强弱的理性判断依据,至今仍有深远意义,然而此类问题的结构复杂,例如式\eqref{eq:fa1}-\eqref{eq:fa2}呈现的柔性设计问题\cite{ostrovsky2002}。
\begin{align}
  \label{eq:fa1}
  \min_{d}\quad & \mathcal{E}_\theta\left( \min_{z} C(\symbfit {y,x,d,z,\theta}) \right)\\
  s.t.\ & \max_{\theta\in\mathbb{T}}\min_{z}\max_{j} f_j(\symbfit {y,x,d,z,\theta}) \le 0 \\
  & h(\symbfit {y,x,d,z,\theta}) = 0 \\
  & \mathbb{T}(\symbfit \delta) = \left\{\symbfit \theta | \symbfit \theta^N - \delta\symbfit \theta^- \le \symbfit \theta \le \symbfit \theta^N + \delta\symbfit \theta^{+} \right\} \label{eq:fa2}
  \end{align}

一般的柔性设计问题转化与求解非常困难,后续研究者发展了大量方法,由于和本文研究方法关系不大,此处不再展开。另一方面,为了降低规划问题的自由度,减少极值问题的嵌套层数从而更方便地处理不确定参数,Acevedo与Pistikopoulos发展了多阶段的随机规划方法\cite{acevedo1998}如式\eqref{eq:sp1}-\eqref{eq:sp2}。

\begin{align}
  \max_{\symbfit y,\symbfit d}\quad & \mathcal{E}_\theta\left( \max_{z}\left\{P(\symbfit{y,x,d,z,\theta})\right\}-C(\symbfit d)-\symbfit{c}^T\symbfit y \right) \label{eq:sp1}\\
  s.t.\ & h(\symbfit{y,x,d,z,\theta}) = 0\\
  & g(\symbfit{y,x,d,z,\theta}) \le 0 \\
  & \symbfit \theta \in \left\{ J(\symbfit \theta), \symbfit \theta^L \le \symbfit \theta \le \symbfit \theta^U  \right\} \label{eq:sp2}
  \end{align}

多阶段随机规划方法中针对不确定场景可以利用情景树分解、采样方法或者直接通过分布计算的方式来获取目标函数的期望,因此具有更强的计算可行性。然而上述方法存在一些问题:首先,基于期望目标或随机规划的方法需要大量的实验数据作为支撑,或者假设不确定参数的概率分布具有已知形式,而这在缺少工业经验的连续合成的概念设计问题中往往不可取,导致问题的保守性不可预估或者过于主观。其次,在本文研究情景下,基于流动微反应器的连续合成设计模型中,往往没有设置控制变量$\symbfit z$的需求,因为以API为代表的高附加值化学品生产的工艺往往遵循严格的监管规程,且为了产品的精确合成往往希望减少来自系统外界的干预。最后,随机规划方法无法保证对于预设的不确定集中任意参数的实现值都满足过程设计的可行性,这是很多问题所力求避免的。因此,上述两种思路并不适用于本文的研究。

鲁棒优化与前述的方法有一定本质区别,其只关心不确定集的具体形式,在求解中保证了鲁棒解在不确定集上任意实现值下都具有一致的最优性,即最差情况下的最优性(Worst-case Optimality)。一般的SRO可以简单地表示为式\eqref{eq:ro1}-\eqref{eq:ro2}的形式\cite{bental2009}。
\begin{align}
    \label{eq:ro1}
    \min_{\symbfit{x}\in\mathbb{R}^n}\max_{\xi\in\mathbb{U}}\quad & f(\symbfit{x},\symbfit p(\xi)) \\
    s.t.\ & h(\symbfit x,\symbfit p(\xi)) = 0\\
    & g(\symbfit x,\symbfit p(\xi)) \le 0 \label{eq:ro2}
  \end{align}

鲁棒优化的框架是清晰而直接的,即可以保证模型做出的最优决策在决策者可预见的不确定场景下均是有效的,这即是所谓鲁棒性(Robustness),具体用优化语言表述为在鲁棒解下不确定参数的实现值对应的优化目标代价不小于其他可行实现值对应的代价。然而,决策者如何构建不确定集成为平衡鲁棒优化解的保守性的重要因素,且在不确定集的形式确定的情况下,鲁棒解仍然是相比其他不确定性优化方法更加保守,即优化目标代价最高。因此,学界为了降低鲁棒优化的保守性发展了可调鲁棒优化,即ARO方法。ARO将一部分决策变量放到不确定集的实现之后在求解,因此这部分决策变量被视为“可调的”(Adjustable),相比原有的SRO中,如式\ref{eq:ro1}中的$\symbfit x$,则被视为“不可调的”(non-adjustable)。ARO最早在线性LP问题中提出的形式如\eqref{eq:aro1}-\eqref{eq:aro2}所示。
\label{section:buqueding}
  \begin{align}
    \min_{\symbfit{u,v,W}}\max_{\xi\in\mathbb{U}}\quad & \symbfit{c(\xi)}^T\binom{\symbfit u}{\symbfit v} \label{eq:aro1}\\
    s.t.\ & \symbfit{Uu}+\symbfit{V}(\symbfit w + \symbfit{W\xi}) \le \symbfit b(\symbfit \xi) \\
    & \symbfit v = \symbfit w + \symbfit{W\xi} \label{eq:aro2}
  \end{align}

问题\eqref{eq:aro1}中$\symbfit u$即为不可调变量,也被称为“当下”(here-and-now)变量;$\symbfit u$即为可调变量,也被称为“观望”(wait-and-see)变量\cite{bental2004}。如今ARO的应用场景不仅仅限制在最简单的LP问题,而是各种过程设计中面临的决策场景。然而,由于一般的ARO问题常常是高度不可解的NP-hard问题,因此在电力能源供应等线性系统的设计场景中应用较为广泛\cite{zugno2016,moreira2015, zhang2016, nan2023},而在化工过程系统综合与优化中的应用较少。

\subsection{两阶段可调鲁棒优化}

正如在\ref{section:buqueding}节中提到的,ARO在最早提出的工作中采用了将可调变量转化为不确定参数的仿射函数的形式,即式\eqref{eq:aro2},类似的数学结构被广泛地研究并拓展为通用的寻找ARO的近似解的方法。类似的方法的核心均为把可调变量表示为某个不确定参数的显式函数,称为“决策规则” (Decision Rule)。决策规则的选取将影响求解的效率和解的性质,并且即使在固定追索(fixed recourse)——即可调变量的系数不含有不确定性——的情形下,引入决策规则的解仍然不是原鲁棒问题的精确解。,为了求解的便利和更强的可解释性,可调鲁棒优化借鉴了多阶段随机规划的思路也形成了两阶段形式,其中第二阶段针对可调变量的最优化问题也被称为追索问题,含有追索问题的ARO的一般形式如式\eqref{eq:tsaro1}-\eqref{eq:tsaro2}所示。
 \begin{align}
   \min_{\symbfit y} &\symbfit c^T\symbfit y + \max_{\symbfit u\in\mathbb U} \min_{\symbfit x\in \mathit F(\symbfit y, \symbfit u)} \symbfit b^T \symbfit x \label{eq:tsaro1}\\
    s.t.\ & f(\symbfit y,\symbfit u) \le 0 \\
    & \mathit F(\symbfit y, \symbfit u) = \left\{ \symbfit x\in \mathbb S_x\right | g(\symbfit x, \symbfit y, \symbfit u) \le 0\} \label{eq:tsaro2}
 \end{align}

在两阶段形式中,“当下”与“观望”变量的关系可以被更清晰地描述,而且更好地与具体问题决策相联系,因此两阶段形式在具体的模型设计中比原始形式更加常用。然而,两阶段ARO的求解必须采用特定的求解流程,而无法采用商用求解器直接求解,且绝大部分ARO被证明是难以计算的\cite{bental2004}。目前,针对两阶段ARO,有一些基于切平面方法的算法流程可以对特定问题进行最优化求解\cite{takeda2008},但是大部分方法尚未发现良好的应用前景,较为广泛应用的方法都在Benders分解\cite{geoffrion1972}的基础上,引入追索问题的对偶鲁棒对等形式(Robust Counterpart, RC),使追索问题成为关于不确定参数的单层优化问题:如下所示,线性两阶段问题\eqref{eq:tsaro1}-\eqref{eq:tsaro2}(令式\eqref{eq:tsaro2}中的$g(\symbfit x, \symbfit y, \symbfit u) \le 0$为$\symbfit G \symbfit x \ge \symbfit h - \symbfit E \symbfit y - \symbfit M \symbfit u$)可转化为由如\eqref{eq:bdmp1}-\eqref{eq:bdmp2}的主问题和如\eqref{eq:bdsp1}-\eqref{eq:bdsp2}的子问题形成的迭代结构进行求解\cite{zeng2013}。
\begin{align}
  \min_{\symbfit y}\quad &\symbfit c^T\symbfit y + \eta \label{eq:bdmp1}\\
   s.t.\ & f(\symbfit y,\theta) \le 0 \\
   & \eta \ge \left(\symbfit h - \symbfit E \symbfit y - \symbfit M \symbfit u^*_l\right)^T\symbfit \pi^*_l,\ \forall 1\le l\le k \label{eq:bdmp2} \\
  \max_{\symbfit u\in\mathbb U, \symbfit \pi}\quad &\left(\symbfit h - \symbfit E \symbfit y - \symbfit M \symbfit u\right)^T\symbfit \pi \label{eq:bdsp1}\\
   s.t.\ & \symbfit G^T\symbfit \pi \le \symbfit b \\
   & \symbfit \pi \ge 0 \label{eq:bdsp2}
\end{align}
此外,针对相同的ARO问题,Zeng和Zhao还提出了一种基于列与约束生成算法的求解流程\cite{zeng2013}。然而,后者目前也只能应用于追索问题的强对偶形式成立的条件,因此无法解决追索问题约束$g(\symbfit x, \symbfit y, \symbfit u) \le 0$含有非凸形式或者混合整数变量的情形。尽管如此,两阶段ARO仍然在工业界的复杂过程设计问题已经崭露头角,显示出方法的优越性与应用前景\cite{gong2017}。

针对追索问题含有混合整数变量的问题,一种可行的思路是将追索变量表示为0-1二元变量,然后利用特殊的分段决策规则,从而将追索问题转化为一个高维MILP问题进行求解\cite{zhang2016}。然而这种方法仍然存在基于决策规则的方法共有的弊端,且计算成本高昂,只适用于追索问题的单变量维度较低,其并非关键的设计决策变量的情况。在精确求解流程方面,Zhao提出了一种基于嵌套的列与约束生成算法的尝试\cite{zhao2012},首次提出了对两阶段问题的三层转化求解思路,对本文所提出的方法启发较大。目前在不确定性下的化工过程设计优化模型中还尚未见到对混合整数追索变量的研究,因此本文所提出的方法对于相关的研究可能有借鉴意义。

\section{论文结构与内容安排}
针对前述部分的研究现状与存在的问题,本文的研究目标总结如下:

第一,建立卢非酰胺连续合成的过程设计模型,结合超结构优化设计与可调鲁棒优化方法,研究一种在多种生产过程不确定性下同时筛选最优合成路径选择与考虑生产过程设计的两阶段模型。

第二,针对两阶段设计模型中存在的混合整数追索问题,提出了一种改进的基于列与约束生成算法的求解流程,应用于卢非酰胺连续合成的过程设计模型的最优化求解,获得兼具鲁棒性与经济效益的最优设计方案。

根据研究目标,本文的结构与内容安排如下:第1章为论文的引言,阐述了论文的研究背景与意义,即高附加值化学品连续化合成的趋势与面临的挑战,随后从连续合成过程设计的超结构优化方法与不确定性优化方法两方面对目前研究取得的主要进展进行了综述,并总结了可调鲁棒优化算法研究中的关键挑战和本文研究的基础列与约束生成算法的相关工作,并阐明了研究问题;第2章为论文的模型建立部分,介绍了针对卢非酰胺连续合成的过程设计优化问题的确定性模型的建立和不确定性优化模型的推导;第3章为论文的求解方法与结果部分,介绍了根据卢非酰胺连续合成可调鲁棒优化设计模型的混合整数追索问题提出的改进的求解流程,并展示了模型的求解结果,提供了不确定性优化的结果分析与鲁棒性检验;第四章为论文结论及展望部分。




